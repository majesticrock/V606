\section{Diskussion}
\label{sec:Diskussion}
Der erste Versuchsteil verlief erwartungsgemäß. Die Durchlassfrequenz
\\ \\
\centerline{$\nu_{\symup{A}} = (35,375 \pm 0,007) \, \symup{kHz}$}
\\ \\
weicht von den Angaben auf dem Gerät 
\\ \\
\centerline{$\nu_{\symup{A}} = 35 \, \symup{kHz}$}
\\ \\ 
um $1,07 \%$ ab, was als sehr gut gewertet werden kann und vor allem durch das Alter des verwendeten Gerätes erklärt werden kann.
Die erechnete Güte von 
\\ \\
\centerline{$Q = 77,8 \pm 1,7 $.}    
\\ \\
weicht um $22,2 \%$ vom eingestellten Wert $Q = 100$ ab. Hier ist vorrangig die Messungenauigkeiten in der Nähe des Maximums zu Betrachten. 
Sowohl der Synthesizer als auch das verwendete Voltmeter waren starken Schwankungen unterworfen, was eine genaue Messung schwierig machte.
Zumal das Ablesen von der analogen Anzeige des Voltmeters immer mit einer Ungenauigkeit behaftet ist.

Im zweiten Versuchsteil sind verschiedene Suszeptibilitäten bestimmt worden. Die experimentell bestimmten Werte werden dabei mit den zuvor
theoretisch bestimmten Werten verglichen. Die Werte sowie die Abweichung vom jeweiligen Theoriewert ist in \autoref{tab:vergleich} zu sehen.
\begin{table}[!htp]
\centering
\caption{Vergleich zwischen den berechneten Suszeptibilitäten und den theoretischen.}
\label{tab:vergleich}
\begin{tabular}{c c c c c c}
\toprule
{Probe} & {$\chi_{\symup{theoretisch}}$}& {$\chi_{\symup{U}}$} &{$\Delta \chi_{\symup{U}}$} & {$\chi_{\symup{R}}$} & {$\Delta \chi_{\symup{R}}$}  \\
\midrule
$\symup{Nd}_2\symup{O}_3$               & $0,003 $   &  $(0,005 \pm 0,005)$   & $70\%$   &  $(0,0021 \pm 0,0004)$   & $ 30\%$ \\
$\symup{Gd}_2\symup{O}_3$               & $0,014 $   &  $(0,25 \pm 0,02)$    & $1710\%$ &  $ (0,0104 \pm 0,0003)$  & $ 25,5\%$ \\
$\symup{Dy}_2\symup{O}_3$               & $0,025 $   &  $(0,70 \pm 0,03)$     & $2710\%$ &  $(0,1882 \pm 0,0003)$   & $ 24,7\%$ \\
$\symup{C}_6\symup{O}_{12}\symup{Pr}_2$ & $0,001 $   &  $(0,0025 \pm 0,0025)$ & $150\%$  &  $(0,00213 \pm 0,00025)$ & $ 113\%$ \\
\bottomrule
\end{tabular}
\end{table}
Die Abweichungen werden nach 
\begin{equation}
\label{eqn:abweich}
\Delta = \Bigl| \frac{\chi_{\symup{theoretisch}} - \chi_{\symup{experimentell}}}{\chi_{\symup{theoretisch}}} \Bigr| \cdot 100
\end{equation}
berechnet. Wie zu sehen ist, sind die Abweichungen alle sehr groß. Dies liegt an den vielen Ungenauigkeiten bei diesem Versuch.
Zum einen sind die Messgeräte nicht genau, das heißt es kann ein großer Ablesefehler vorliegen, zumal wie auch schon im Teil zuvor
die Nadel am Voltmeter geschwankt hat. Des Weiteren konnte die zuvor bestimmte Durchlassfrequenz nicht genau eingestellt werden, da die 
Skala am Generator zu grob ist. Hinzukommt, dass sich die Brückenspannung nicht genau auf 0 abgleichen ließ. Auch konnte der 
Innenwiderstand des Verstärkers nicht berücksichtigt werden, was ebenfalls eine Verfälschung darstellt. Die Suszeptibilitäten der einzelnen 
Stoffe sind zudem temperaturabhängig, weshalb diese nicht lange in der Hand oder in der Spule gehalten werden konnten. Die Ergebnisse 
der Durchführung waren an sich nie rekonstruierbar, weshalb diese eher als Zufall beschrieben werden können. Es ist daher nicht möglich
auf Grund dieser Ergebnisse physikalisch korrekte Aussagen zu treffen, da lediglich die Bestimmung über die Widerstände vertretbare 
Abweichungen aufweist (abgesehen von $\symup{C}_6\symup{O}_{12}\symup{Pr}_2$), was ersteinmal verständlich erscheint, da die Apparatur
zur Bestimmung er Widerstände bis auf $5 \, \symup{m\Omega}$ genau ist. Da aber auch diese Messung mit dem selben Aufbau durchgeführt wurde
kann es sich heirbei auch um einen Zufall handeln. 