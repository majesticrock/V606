\section{Auswertung}
\label{sec:Auswertung}
    \subsection{Untersuchung der Filterkurve}
    In diesem Teil des Versuches wird zunächst die Durchlassfrequenz $\nu$ des Selektivverstärkers bestimmt, wobei eine
    Eingangsspannung $U_{\symup{E}} = 100$ mV verwendet wird. Die gemessene Zuordnung
    der Frequenzen $\nu$ mit den Spannungen $U_{\symup{A}}$ ist in \autoref{tab:selektiv} zu sehen. 
    Die Spannungen werden gegen die Frequenzen aufgetragen, was in \autoref{fig:selektiv} zu sehen ist. Außerdem wird eine Lorentzkurve
    \begin{equation}
    \label{eqn:lorentz}
        U = \frac{a}{(\nu^2 - \nu_{\symup{A}}^2)^2 + b^2 \nu_{\symup{A}}^2}
    \end{equation}
    durch die Messwerte von $34$ kHz bis $36$ kHz mittels des Python 3.7.0 Paketes Curve Fit gelegt. Die Parameter ergeben dadurch sich zu 
    \begin{align*}
    \nu_{\symup{A}} &= (35,375 \pm 0,007) \, \symup{kHz}, \\
    a &= (6,9 \pm 0,3) \cdot 10^4 \, \symup{\frac{V}{s^4}}, \\
    b &= (0,71 \pm 0,02) \, \symup{\frac{1}{s^2}}.
    \end{align*}
    Des Weiteren wird eine Gerade bei $\frac{1}{\sqrt{2}}$ der Maximalspannung $U_{\symup{max}}$ gelegt, also
    $\frac{U_{\symup{max}}}{\sqrt{2}} = 77.78 \, \symup{V}$, gelegt. Die Schnittpunkte mit der Lorentzkurve sind bei
    Frequenzen
    \begin{align*}
    \nu_{\symup{+}} &=
    \nu_{\symup{-}} &=
    \end{align*} 
    gegeben. Dadurch ergibt sich die Güte
    \begin{equation}
    \label{eqn:guete}
        Q = \frac{\nu_{\symup{A}}}{\nu_{\symup{-}} - \nu_{\symup{+}}}
    \end{equation}
    zu 
    \\ \\
    \centerline{$Q = $.}    
    \\ \\
    Der Fehler erechnet sich dabei über
    \\ \\
    \centerline{$\sqrt{\frac{\sigma_{\symup{\nu_{1}}}^{2} \nu_{\symup{A}}^{2}}{\left(\nu_{1} - \nu_{2}\right)^{4}} + \frac{\sigma_{\symup{\nu_{2}}}^{2} \nu_{\symup{A}}^{2}}{\left(\nu_{1} - \nu_{2}\right)^{4}} + \frac{\sigma_{\symup{\nu_{\symup{A}}}}^{2}}{\left(\nu_{1} - \nu_{2}\right)^{2}}}$}
    \\ \\
    \begin{table}[!htp]
\centering
\caption{Messwerte der Filterkurve des Selektivverstärkers.}
\label{tab:selektiv}
\begin{tabular}{c c}
\toprule
{Frequenz $\nu$ / kHz} & {Spannung $U$ / mV} \\
\midrule
20.0 & 0.85 \\
21.0 & 0.95 \\
22.0 & 1.10 \\
23.0 & 1.25 \\
24.0 & 1.40 \\
25.0 & 1.60 \\
26.0 & 1.85 \\
27.0 & 2.15 \\
28.0 & 2.55 \\
29.0 & 3.05 \\
30.0 & 3.30 \\
31.0 & 4.20 \\
32.0 & 5.80 \\
33.0 & 8.90 \\
34.0 & 10.50 \\
34.1 & 11.00 \\
34.2 & 12.00 \\
34.3 & 13.00 \\
34.4 & 17.00 \\
34.5 & 19.00 \\
34.6 & 22.00 \\
34.7 & 26.00 \\
34.8 & 34.00 \\
34.9 & 38.00 \\
35.0 & 42.00 \\
35.1 & 62.00 \\
35.2 & 88.00 \\
35.3 & 105.00 \\
35.4 & 110.00 \\
35.5 & 90.00 \\
35.6 & 76.00 \\
35.7 & 60.00 \\
35.8 & 41.00 \\
35.9 & 33.00 \\
36.0 & 26.00 \\
37.0 & 11.00 \\
38.0 & 7.80 \\
39.0 & 5.70 \\
40.0 & 4.40 \\
\bottomrule
\end{tabular}
\end{table} 
    \begin{figure}
        \centering
        \includegraphics{selektiv.pdf}
        \caption{Plot der gemessenen Spannungen gegen die Frequenz mit Lorentzkurvenfit und Gerade zur Bestimmung der .}
    \label{fig:selektiv}
    \end{figure}

    \subsection{Theoretische Bestimmung der Suszeptibilitäten}
    Die Suszeptibilitäten der hier betrachteten Stoffe lassen sich nach Gleichung \eqref{eqn:Suszep???} bestimmen.
    Der benötigte Lande-Faktor $g_{\symup{J}}$, sowie die Größen $L$, $S$ und $J$ der verschiedenen Stoffe sind in \autoref{tab:stoff}
    aufgeführt. Des Weiteren zu sehen sind darin die Dichte $\rho$, die Masse $m$, die zugehörige molare Masse $M$ und die Zahl der magnetischen
    Momente pro Volumeneinheit $N$, welche sich nach 
    \begin{equation}
    \label{eqn:N}
        N = 2 \cdot \frac{\rho}{M} N_{\symup{A}}
    \end{equation}
    berechnet, wobei $N_{\symup{A}}$ die Avogadrokonstante ist. Die Werte von $J$, $L$ und $S$ werden durch Betrachtung der 4f-Elektronen 
    der Stoffe bestimmt. Dabei ist zu Beachten, dass je nach Anzahl dieser Elektronen die Möglichkeiten die Spins anzuordnen variieren.
    Nach dem Pauli-Prinzip sollen diese maximiert werden, wobei zu Beachten ist, dass es die Qunatenmechanik nicht zulässt, dass zwei 
    Elektronen eines Atoms nicht in allen Quantenzahlen übereinstimmen. Dies wird beispielhaft für $\symup{Gd}_2\symup{O}_3$ gezeigt.
    Dieser Stoff hat 7 4f Elektronen. Nach der Hundschen Regel ergibt sich $L = 0$. Der Wert für $S$ ergibt sich außerdem zu 
    $S = 3,5$. Damit bstimmt sich $J$ zu $J = L - S = 3,5$. 