\section{Auswertung}
\label{sec:Auswertung}
    \subsection{Untersuchung der Filterkurve}
    In diesem Teil des Versuches wird zunächst die Durchlassfrequenz $\nu_{\symup{A}}$ des Selektivverstärkers bestimmt, wobei eine
    Eingangsspannung $U_{\symup{E}} = 100$ mV verwendet wird. Es sei außerdem erwähnt, dass hier mit einer Verstärkung um den Faktor 100
    gearbeitet wird. Die gemessene Zuordnung
    der Frequenzen $\nu$ mit den Spannungen $U$ ist in \autoref{tab:selektiv} zu sehen. 
    Die Spannungen werden gegen die Frequenzen aufgetragen, was in \autoref{fig:selektiv} zu sehen ist. Außerdem wird eine Lorentzkurve
    \begin{equation}
    \label{eqn:lorentz}
        U = \frac{a}{(\nu^2 - \nu_{\symup{A}}^2)^2 + b^2 \nu_{\symup{A}}^2}
    \end{equation}
    durch die Messwerte von $34$ kHz bis $36$ kHz mittels des Python 3.7.0 Paketes Curve Fit gelegt. Die Parameter ergeben sich dadurch zu 
    \begin{align*}
    \nu_{\symup{A}} &= (35,375 \pm 0,007) \, \symup{kHz}, \\
    a &= (6,9 \pm 0,3) \cdot 10^4 \, \symup{\frac{V}{s^4}}, \\
    b &= (- 0,71 \pm 0,02) \, \symup{\frac{1}{s^2}}.
    \end{align*}
    Des Weiteren wird eine Gerade bei $\frac{1}{\sqrt{2}}$ der Maximalspannung $U_{\symup{max}} = (107 \pm 9)$ V gelegt. Die Schnittpunkte mit der Lorentzkurve sind bei den
    Frequenzen
    \begin{align*}
    \nu_{\symup{+}} &= (35,146 \pm 0,007) \symup{kHz}\\
    \nu_{\symup{-}} &= (35,601 \pm 0,007) \symup{kHz}
    \end{align*} 
    gegeben. Dadurch ergibt sich die Güte
    \begin{equation}
    \label{eqn:guete}
        Q = \frac{\nu_{\symup{A}}}{\nu_{\symup{-}} - \nu_{\symup{+}}}
    \end{equation}
    zu 
    \\ \\
    \centerline{$Q = 77,8 \pm 1,7 $.}    
    \\ \\
    \begin{table}[!htp]
\centering
\caption{Messwerte der Filterkurve des Selektivverstärkers.}
\label{tab:selektiv}
\begin{tabular}{c c}
\toprule
{Frequenz $\nu$ / kHz} & {Spannung $U$ / mV} \\
\midrule
20.0 & 0.85 \\
21.0 & 0.95 \\
22.0 & 1.10 \\
23.0 & 1.25 \\
24.0 & 1.40 \\
25.0 & 1.60 \\
26.0 & 1.85 \\
27.0 & 2.15 \\
28.0 & 2.55 \\
29.0 & 3.05 \\
30.0 & 3.30 \\
31.0 & 4.20 \\
32.0 & 5.80 \\
33.0 & 8.90 \\
34.0 & 10.50 \\
34.1 & 11.00 \\
34.2 & 12.00 \\
34.3 & 13.00 \\
34.4 & 17.00 \\
34.5 & 19.00 \\
34.6 & 22.00 \\
34.7 & 26.00 \\
34.8 & 34.00 \\
34.9 & 38.00 \\
35.0 & 42.00 \\
35.1 & 62.00 \\
35.2 & 88.00 \\
35.3 & 105.00 \\
35.4 & 110.00 \\
35.5 & 90.00 \\
35.6 & 76.00 \\
35.7 & 60.00 \\
35.8 & 41.00 \\
35.9 & 33.00 \\
36.0 & 26.00 \\
37.0 & 11.00 \\
38.0 & 7.80 \\
39.0 & 5.70 \\
40.0 & 4.40 \\
\bottomrule
\end{tabular}
\end{table} 
    \begin{figure}
        \centering
        \includegraphics{selektiv.pdf}
        \caption{Plot der gemessenen Spannungen gegen die Frequenz mit Lorentzkurvenfit und Gerade zur Bestimmung der Durchlassfrequenz und der Güte.}
    \label{fig:selektiv}
    \end{figure}

    \subsection{Theoretische Bestimmung der Suszeptibilitäten}
    Die Suszeptibilitäten der hier betrachteten Stoffe lassen sich nach Gleichung \eqref{eqn:suszep} bestimmen.
    Der benötigte Lande-Faktor $g_{\symup{J}}$, sowie die Größen $L$, $S$ und $J$ der verschiedenen Stoffe sind in \autoref{tab:theoriewerte}
    aufgeführt. Des Weiteren zu sehen sind darin die Dichte $\rho$, die Masse $m$, die zugehörige molare Masse $M$ und die Zahl der magnetischen
    Momente pro Volumeneinheit $N$, welche sich nach 
    \begin{equation}
    \label{eqn:N}
        N = 2 \cdot \frac{\rho}{M} N_{\symup{A}}
    \end{equation}
    berechnet, wobei $N_{\symup{A}}$ die Avogadrokonstante ist. Die Werte von $J$, $L$ und $S$ werden durch Betrachtung der 4f-Elektronen 
    der Stoffe bestimmt. Dabei ist zu Beachten, dass je nach Anzahl dieser Elektronen die Möglichkeiten die Spins anzuordnen variieren.
    Nach dem Pauli-Prinzip sollen diese maximiert werden, wobei zu Beachten ist, dass es die Qunatenmechanik nicht zulässt, dass zwei 
    Elektronen eines Atoms nicht in allen Quantenzahlen übereinstimmen. Dies wird beispielhaft für $\symup{Dy}_2\symup{O}_3$ gezeigt.
    Dieser Stoff hat 9 4f Elektronen. Nach der Hundschen Regel ergibt sich $L = -3-2-1-0+1 = -5$, wobei der Betrag betrachtet wird.
    Der Wert für $S$ ergibt sich außerdem zu $S = 7 \cdot \frac{1}{2} - 2 \cdot \frac{1}{2} = 2,5$.
     Damit bstimmt sich $J$ zu $J = L - S = 4,5$. Diese Werte sowie die Dichten $\rho$, die Massen $m$, molaren Massen $M$ und die 
    Teilchenanzahl pro Volumeneinheit $N$ sind in \autoref{tab:theoriewerte} zu sehen. Außerdem ist der Landefaktor dargestellt,
    welcher sich nach Gleichung \eqref{eqn:lande} berechnet. In dieser Rechnung wird eine Raumtemperatur von $T = 293,15$ K angenommen.

    \begin{table}[!htp]
\centering
\caption{Materialeigenschaften der Seltenen Erd Verbindungen zur Bestimmung der Theoriewerte der Suszeptibilität.}
\label{tab:theoriewerte}
\begin{tabular}{c c c c c}
\toprule
{} & {$\symup{Nd}_2\symup{O}_3$} & {$\symup{Gd}_2\symup{O}_3$} & {$\symup{Dy}_2\symup{O}_3$} & {$\symup{C}_6\symup{O}_{12}\symup{Pr}_2$} \\
\midrule
4f-Elektronen & 3 & 7 & 9 & 3 \\
$L$ & 6 & 0 & 5 & 5 \\
$S$ & 1,5 & 3,5 & 2,5 & 1 \\ 
$J$ & 4,5 & 3,5 &7.5 & 4 \\
$g_{\symup{J}}$ & 0,73 & 2,0 & 1,33 & 0,73 \\
$\rho$ / $\frac{\symup{kg}}{\symup{m^3}}$ & 7240 & 7400 & 7800 & 6260 \\
$m$ / kg & 0,009 & 0,0141 & 0,0151 & 0,0079 \\
$M$ / $\symup{\frac{g}{mol}}$ & 336 & 362 & 373 & 544 \\ 
$N$ / $10^{28} \frac{1}{\symup{m^3}}$ & 2,59 & 2,46 & 2,52 & 1,38 \\
\bottomrule
\end{tabular}
\end{table}
    Durch Anwenden von Gleichung \eqref{eqn:suszep}, unter Verwendung der Werte aus \autoref{tab:theoriewerte}, ergeben sich die gesuchten 
    Suszeptibilitäten der verschiednen Stoffe zu 
    \begin{align*}
        \chi_{\symup{Nd}_2\symup{O}_3} &= 0,003, \\
        \chi_{\symup{Gd}_2\symup{O}_3} &= 0,014, \\
        \chi_{\symup{Dy}_2\symup{O}_3} &= 0,025, \\
        \chi_{\symup{C}_6\symup{O}_{12}\symup{Pr}_2} &= 0,001. 
    \end{align*} 

    \subsection{Experimentelle Bestimmung der Suszeptibilitäten}
    Die experimentelle Bestimmung der Suszeptibilitäten verläuft nach Gleichung \eqref{eqn:suszep-spannung} und \eqref{eqn:suszep-wider},
    wobei das dort verwendete $Q$ durch den Querschnitt
    \begin{equation}
    \label{eqn:qreal}
    Q_{\symup{real}} = \frac{m}{\rho l}
    \end{equation}
    ersetzt wird. Dieser entspricht dem Querschnitt, welcher aufträte wenn die Probe nur aus einem Einkristall bestünde.
    Hierbei ist $m$ die Masse der Probe, $\rho$ die Dichte der Probe und $l$ die Länge der Probe. Letztere beträgt bei allen 
    Proben $l = 135$ mm. Masse und Dichte der Proben sind in \autoref{tab:theoriewerte} zu sehen.
    Die berechneten Querschnitte sind in \autoref{tab:querschnitt} zu sehen.
    \begin{table}[!htp]
\centering
\caption{Querschnitte der untersuchten Proben.}
\label{tab:querschnitt}
\begin{tabular}{c c}
\toprule
{Probe} & {$Q_{\symup{real}} / 10^{-5} m^2$}  \\
\midrule
$\symup{Nd}_2\symup{O}_3$               &  0.921 \\  
$\symup{Gd}_2\symup{O}_3$               & 1.411 \\
$\symup{Dy}_2\symup{O}_3$               & 1.434 \\
$\symup{C}_6\symup{O}_{12}\symup{Pr}_2$ & 0.935 \\
\bottomrule
\end{tabular}
\end{table}
    Des Weiteren wird eine Eingangsspannung von $U_{\symup{Speis}} = 0,5$ V verwendet und wie schon im Versuchsteil zuvor eine Gesamtverstärkung
    des Signals um den Faktor 100. In \autoref{tab:Nd2O3}, 5, 6 und 7 sind die gemessenen
    Werte zu sehen. Hierbei bezeichnet $U_0$ beziehungsweise $R_0$ die anfangs gemessene Spannung beziehungsweise Widerstand, also ohne
    Probe in der Spule, während $U_{\symup{P}}$ und $R_{\symup{P}}$ die gemessenen Werte mit in die Spule eingeführter Probe sind. Des
    Weiteren sind die Differenzen $\Delta R$ und $\Delta U$ ebenfalls dargestellt.
    \begin{table}[!htp]
\centering
\caption{Messwerte und zugehörige Differenzen von $\symup{Nd}_2\symup{O}_3$ .}
\label{tab:Nd2O3}
\begin{tabular}{c c c c c c}
\toprule
{$U_0$ / mV} & {$R_0$ / $\symup{\Omega}$} & {$U_{\symup{P}}$ / mV} & {$R_{\symup{P}}$ / $\symup{\Omega}$} & {$\Delta R$ / $\symup{\Omega}$} & {$\Delta U$ / mV}\\
\midrule
8.1 & 2.600 & 8.1 & 2.535 & 0.065 & 0 \\
8.1 & 2.650 & 8.3 & 2.520 & 0.13 & 0.2 \\
8.1 & 2.630 & 8.1 & 2.490 & 0.14 & 0 \\
\bottomrule
\end{tabular}
\end{table}
    \begin{table}[!htp]
\centering
\caption{Messwerte und zugehörige Differenzen von  $\symup{Gd}_2\symup{O}_3$.}
\label{tab:Gd2O3}
\begin{tabular}{c c c c c c}
\toprule
{$U_0$ / mV} & {$R_0$ / $\symup{\omega}$} & {$U_{\symup{P}}$ / mV} & {$R_{\symup{P}}$ / $\symup{\omega}$} & {$\Delta R$ / $\symup{\omega}$} & {$\Delta U$ / mV} \\
\midrule
6.5 & 2.655 & 12.0 & 1.780 & 0.875 & 5.5 \\
7.0 & 2.605 & 11.5 & 1.800 & 0.805 & 4.5 \\
6.5 & 2.635 & 12.0 & 1.770 & 0.865 & 5.5 \\
\bottomrule
\end{tabular}
\end{table}
    \begin{table}[!htp]
\centering
\caption{Messwerte und zugehörige Differenzen von $\symup{Dy}_2\symup{O}_3$.}
\label{tab:Dy2O3}
\begin{tabular}{c c c c c c}
\toprule
{$U_0$ / mV} & {$R_0$ / $\symup{\Omega}$} & {$U_{\symup{P}}$ / mV} & {$R_{\symup{P}}$ / $\symup{\Omega}$} & {$\Delta R$ / $\symup{\Omega}$} & {$\Delta U$ / mV} \\
\midrule
7.9 & 2.565 & 21.5 & 1.055 & 1.510 & 13.6 \\
6.5 & 2.605 & 21.0 & 1.055 & 1.550 & 14.5 \\
6.5 & 2.635 & 22.0 & 1.030 & 1.605 & 15.5 \\
\bottomrule
\end{tabular}
\end{table}
    \begin{table}[!htp]
\centering
\caption{Messwerte und zugehörige Differenzen von $\symup{C}_6\symup{O}_{12}\symup{Pr}_2$.}
\label{tab:C6O12Pr2}
\begin{tabular}{c c c c c c}
\toprule
{$U_0$ / mV} & {$R_0$ / $\symup{\Omega}$} & {$U_{\symup{P}}$ / mV} & {$R_{\symup{P}}$ / $\symup{\Omega}$}  & {$\Delta R$ / $\symup{\Omega}$} & {$\Delta U$ / mV}\\
\midrule
8.2 & 2.665 & 8.1 & 2.570 & 0.095 & 0.1 \\
8.0 & 2.650 & 8.0 & 2.540 & 0.110 & 0 \\
8.0 & 2.670 & 8.0 & 2.530 & 0.140 & 0 \\
\bottomrule
\end{tabular}
\end{table}
    Über diese Differenzen wird mit Python 3.7.0 gemittelt. Die gemittelten Werte sind in \autoref{tab:mittel} zu sehen.
    \begin{table}[!htp]
\centering
\caption{Gemittelte Spannungs- und Widerstandsdifferenzen.}
\label{tab:mittel}
\begin{tabular}{c c c}
\toprule
{Probe} & {$\Delta \overline{U}$ / mV} & {$\Delta \overline{R}$ / $\symup{\Omega}$} \\
\midrule
$\symup{Nd}_2\symup{O}_3$               & $0.07 \pm 0.07$ & $ 0.11 \pm 0.02 $ \\
$\symup{Gd}_2\symup{O}_3$               & $5.2 \pm 0.3  $ & $ 0.85 \pm 0.02$ \\
$\symup{Dy}_2\symup{O}_3$               & $14.5 \pm 0.5 $ & $ 1.55 \pm 0.03 $\\
$\symup{C}_6\symup{O}_{12}\symup{Pr}_2$ & $0.03\pm 0.03 $ & $ 0.11 \pm 0.01$\\
\bottomrule
\end{tabular}
\end{table}
    Nach Gleichung \eqref{eqn:suszep-spannung} und \eqref{eqn:suszep-wider} lässt sich die Suszeptibilität einmal über den Widerstand und 
    einmal über die Spannung bestimmen. Dabei werden die gemittelten Werte aus \autoref{tab:mittel} verwendet.
    Des Weiteren ist $F = 86,6 \symup{mm}^2$ der Querschnitt der Spule und $R_3 = 998 \, \symup{ \Omega}$ ein Ohmscher Widerstand.
    Die erechneten Suszeptibilitäten sind in \autoref{eqn:suszep} dargestellt.
    \begin{table}[!htp]
\centering
\caption{Berechnete Suszeptibilitäten.}
\label{tab:suszep}
\begin{tabular}{c c c}
\toprule
{Probe} & {$\chi_{\symup{U}}$} & {$\chi_{\symup{R}}$} \\
\midrule
$\symup{Nd}_2\symup{O}_3$               & $(0,005 \pm 0,005)$   &    $(0,0021 \pm 0,0004)$  \\
$\symup{Gd}_2\symup{O}_3 $              & $(0,25 \pm 0,02)$    &    $ (0,0104 \pm 0,0003)$ \\
$\symup{Dy}_2\symup{O}_3$               & $(0,70 \pm 0,03)$     &    $(0,1882 \pm 0,0003)$  \\
$\symup{C}_6\symup{O}_{12}\symup{Pr}_2$ & $(0,0025 \pm 0,0025)$ &    $(0,00213 \pm 0,00025)$\\
\bottomrule
\end{tabular}
\end{table}